\section{Broader Impacts}

Shortcutting as developed in this proposal can provide significant performance improvements for sub-optimal or approximate calculations.  This covers an extremely broad range of real-world problems from optimization to path planning, and from convex programming to sub-optimal NP-hard problems.  Shortcutting will thus have an profoundly broad impact right from the start. One particular area that will be strongly impacted is medical imaging and intensity modulated particle therapy, which is an application area shared by Drs. Gomez, Schubert, and Censor in collaboration with Dr. Reinhard Schulte from Loma Linda University Medical Center (LLUMC), whose letter of support is attached to this proposal.  As a result there is an immediate application area, and connection to other experts in the field.

The results will be sent to the appropriate area journals of IEEE, SIAM, ACM, and other high-quality journals in the field, to ensure maximum dissemination.  Papers and data will be made available on the web in repositories such as arXiv to maximize readership.

In this research we will be concentrating on development of shortcutting for sub-optimal and approximate computation, however shortcutting could, in theory, also be used for exact solutions.  This opens up another wide class of problems which can be handled by shortcutting.


The Baylor - Cal State collaboration not only brings together excellence in research, but also impacts on traditionally underrepresented groups.  California State University, San Bernardino is an Hispanic Serving Institution, and as Dr. Gomez is himself a member of this group, he is uniquely qualified to support and mentor students from underrepresented groups.  Additionally, Baylor University has a record of inclusiveness of women, in that 18\% of the engineering school students are women, as opposed to 12\% of engineering school students are women nationally.  Dr. Schubert's research group takes pride in having half of the group members female.  The M.S. students from CSUSB that are on this proposal will be encouraged to come to Baylor University and continue their participation in this research, with Dr. Gomez on their committee.  This will be used to help establish an integrated support pipeline to encourage students from underrepresented groups to obtain a Ph.D. as they would already have an advisor, committee member, support, and area of research they were interested in.  They would also already know Baylor University, the research environment, and the research group they would be in, which would remove many of the barriers that keep students from underrepresented groups from continuing on in their education.  Thus between the two programs this proposal will also help expand underrepresented groups in computing.
