\section*{Research strategy}

\instructions{ Organize the Research Strategy in the specified order
    and using the instructions provided below. Start each section with
    the appropriate section heading---Significance, Innovation,
    Approach. Cite published experimental details in the Research
    Strategy section and provide the full reference in the
    Bibliography and References cited section.}

%%%%%%%%%%%%%%%%%%%%%%%%%%%%%%%%%%%%%%%%%%%%%%%%%%%%%%%%%%%%%%%%%%%%%
\subsection*{Significance}

\instructions{ Instructions: Explain the importance of the problem or
    critical barrier to progress in the field that the proposed
    project addresses.  Explain how the proposed project will improve
    scientific knowledge, technical capability, and/or clinical
    practice in one or more broad fields.  Describe how the concepts,
    methods, technologies, treatments, services, or preventative
    interventions that drive this field will be changed if the
    proposed aims are achieved.}

\instructions{ Review criteria: Does the project address an important
    problem or a critical barrier to progress in the field? If the
    aims of the project are achieved, how will scientific knowledge,
    technical capability, and/or clinical practice be improved? How
    will successful completion of the aims change the concepts,
    methods, technologies, treatments, services, or preventative
    interventions that drive this field?}

\instructions{Explain the importance of the problem or critical barrier to progress in the field that the proposed project addresses.}

\instructions{Explain how the proposed project will improve scientific knowledge, technical capability, and/or clinical practice in one or more broad fields.}

\instructions{Describe how the concepts, methods, technologies, treatments, services, or preventative interventions that drive this field will be changed if the proposed aims are achieved.}






%%%%%%%%%%%%%%%%%%%%%%%%%%%%%%%%%%%%%%%%%%%%%%%%%%%%%%%%%%%%%%%%%%%%%
\subsection*{Innovation}

\instructions{ Instructions: Explain how the application challenges and
    seeks to shift current research or clinical practice paradigms.
    Describe any novel theoretical concepts, approaches or
    methodologies, instrumentation or intervention(s) to be developed
    or used, and any advantage over existing methodologies,
    instrumentation or intervention(s).  Explain any refinements,
    improvements, or new applications of theoretical concepts,
    approaches or methodologies, instrumentation or interventions.}

\instructions{ Review criteria: Does the application challenge and seek
    to shift current research or clinical practice paradigms by
    utilizing novel theoretical concepts, approaches or methodologies,
    instrumentation, or interventions? Are the concepts, approaches or
    methodologies, instrumentation, or interventions novel to one
    field of research or novel in a broad sense? Is a refinement,
    improvement, or new application of theoretical concepts,
    approaches or methodologies, instrumentation, or interventions
    proposed? }

\instructions{ Explain how the application challenges and seeks to shift current research or clinical practice paradigms.}

\instructions{ Describe any novel theoretical concepts, approaches or methodologies, instrumentation or interventions to be developed }

\instructions{ Explain any refinements, improvements, or new applications of theoretical concepts, approaches or methodologies, instrumentation, or interventions.}





%%%%%%%%%%%%%%%%%%%%%%%%%%%%%%%%%%%%%%%%%%%%%%%%%%%%%%%%%%%%%%%%%%%%%
\subsection*{Approach}

\instructions{ Instructions: Describe the overall strategy,
    methodology, and analyses to be used to accomplish the specific
    aims of the project. Unless addressed separately in Item 5.5.15,
    include how the data will be collected, analyzed, and interpreted
    as well as any resource sharing plans as appropriate.  Discuss
    potential problems, alternative strategies, and benchmarks for
    success anticipated to achieve the aims.  If the project is in the
    early stages of development, describe any strategy to establish
    feasibility, and address the management of any high risk aspects
    of the proposed work.}

\instructions {For new applications, include information on
  Preliminary Studies as part of the Approach section. Discuss the
  PD/PI'ェs preliminary studies, data, and/or experience pertinent to
  this application. Except for Exploratory/Development Grants
  (R21/R33), Small Research Grants (R03), Academic Research
  Enhancement Award (AREA) Grants (R15), and Phase I Small Business
  Research Grants (R41/R43), preliminary data can be an essential part
  of a research grant application and help to establish the likelihood
  of success of the proposed project.}

\instructions{ Review criteria: Are the overall strategy, methodology,
    and analyses well-reasoned and appropriate to accomplish the
    specific aims of the project? Are potential problems, alternative
    strategies, and benchmarks for success presented? If the project
    is in the early stages of development, will the strategy establish
    feasibility and will particularly risky aspects be managed?}

\instructions{ Describe the overall strategy, methodology, and analyses to be used to accomplish the specific aims of the project. Unless addressed separately in Item 15 (Resource Sharing Plan), include how the data will be collected, analyzed, and interpreted as well as any resource sharing plans as appropriate.}

\instructions{ Discuss potential problems, alternative strategies, and benchmarks for success anticipated to achieve the aims.}

\instructions{ If the project is in the early stages of development, describe any strategy to establish feasibility, and address the management of any high risk aspects of the proposed work.}

\instructions{ Point out any procedures, situations, or materials that may be hazardous to personnel and precautions to be exercised. A full discussion on the use of Select Agents should appear in Item 11, below.}

\instructions{ As applicable, also include the following information as part of the Research Strategy, keeping within the three sections listed above: Significance, Innovation, and Approach.}

%\subsubsection*{Preliminary studies}

\instructions{ Preliminary Studies for New Applications: For new applications, include information on Preliminary Studies. Discuss the PD/PI's preliminary studies, data, and or experience pertinent to this application. Except for Exploratory/Developmental Grants (R21/R33), Small Research Grants (R03), and Academic Research Enhancement Award (AREA) Grants (R15), preliminary data can be an essential part of a research grant application and help to establish the likelihood of success of the proposed project. Early Stage Investigators should include preliminary data (however, for R01 applications, reviewers will be instructed to place less emphasis on the preliminary data in application from Early Stage Investigators than on the preliminary data in applications from more established investigators).}


\subsubsection*{Hypothesis}
 
 
 

\subsubsection*{Specific Aim 1:\AIMI}





\textbf{Potential Pitfalls.}




\subsubsection*{Specific Aim 2:\AIMII}




\textbf{Potential Pitfalls and Alternative Approaches.}



\subsubsection*{Specific Aim 3:\AIMIII}



\textbf{Potential Pitfalls and Alternative Approaches.}







%%%%%%%%%%%%%%%%%%%%%%%%%%%%%%%%%%%%%%%%%%%%%%%%%%%%%%%%%%%%%%%%%%%%%%%%%%%%%%%%%%%%%%%%%%%%%%%%%
%   Samples - figure
%%%%%%%%%%%%%%%%%%%%%%%%%%%%%%%%%%%%%%%%%%%%%%%%%%%%%%%%%%%%%%%%%%%%%%%%%%%%%%%%%%%%%%%%%%%%%%%%%
%\begin{figure}[b c] % Centered big figure at bottom of the page ([b] argument, could be ``t'' for top or ``h'' for here)
%\centering
%\includegraphics[scale = .80]{Figures/Fig2.pdf}
%\caption{\footnotesize Big Figure legend Big Figure legend Big Figure legend Big Figure legend Big Figure legend Big Figure legend Big Figure legend Big Figure legend Big Figure legend.}
%\label{fig2}
%\end{figure}
%%%%%%%%%%%%%%%%%%%%%%%%%%%%%%%%%%%%%%%%%%%%%%%%%%%%%%%%%%%%%%%%%%%%%%%%%%%%%%%%%%%%%%%%%%%%%%%%%
%   Samples - wrapfigure
%%%%%%%%%%%%%%%%%%%%%%%%%%%%%%%%%%%%%%%%%%%%%%%%%%%%%%%%%%%%%%%%%%%%%%%%%%%%%%%%%%%%%%%%%%%%%%%%%
%\begin{wrapfigure}{r}{6.8cm} % Example figure with text wrapping around it
%\includegraphics[scale=0.9]{Figures/Fig1.pdf}
%\caption{\footnotesize Example wrapped figure. (A) Impressive microscopy image. (B) Impressive data.}
%\end{wrapfigure}
%%%%%%%%%%%%%%%%%%%%%%%%%%%%%%%%%%%%%%%%%%%%%%%%%%%%%%%%%%%%%%%%%%%%%%%%%%%%%%%%%%%%%%%%%%%%%%%%%
%   Samples - wraptable
%%%%%%%%%%%%%%%%%%%%%%%%%%%%%%%%%%%%%%%%%%%%%%%%%%%%%%%%%%%%%%%%%%%%%%%%%%%%%%%%%%%%%%%%%%%%%%%%%
%\begin{wraptable}{r}{4.8cm} % Example table with text wrapping around it
%\caption{Example Table}
%\begin{center}
%\begin{tabular}{l l r}
%\toprule
%\multicolumn{1}{c}{City} & {N\textsuperscript{a}} & {\%Silly}\\
%\midrule
%San Diego & 289 & 41\%\\
%Seattle & 262 & 32\%\\
%Galveston & 261 & 15\%\\
%St Louis & 269 & 7\%\\
%New York & 271 & 4\%\\
%Baltimore & 231 & 2\%\\
%\emph{Total} & 1,586 & 21\%\\
%\hline
%\end{tabular}\\
%\footnotesize\textsuperscript{a}{All participants clowns.}
%\end{center}
%\label{default}
%\end{wraptable} 