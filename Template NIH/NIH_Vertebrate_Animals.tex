\section*{10. Vertebrate Animals}

\instructions{ If Vertebrate Animals are involved in the project, address each of the five points below. This section should be a concise, complete description of the animals and proposed procedures. While additional details may be included in the Research Strategy, the responses to the five required points below must be cohesive and include sufficient detail to allow evaluation by peer reviewers and NIH staff. If all or part of the proposed research involving vertebrate animals will take place at alternate sites (such as project/performance or collaborating site(s)), identify those sites and describe the activities at those locations. Although no specific page limitation applies to this section of the application, be succinct. Failure to address the following five points will result in the application being designated as incomplete and will be grounds for the PHS to defer the application from the peer review round. Alternatively, the application’s impact/priority score may be negatively affected.}

\instructions{ If the involvement of animals is indefinite, provide an explanation and indicate when it is anticipated that animals will be used. If an award is made, prior to the involvement of animals the grantee must submit to the NIH awarding office detailed information as required in points 1-5 above and verification of IACUC approval. If the grantee does not have an Animal Welfare Assurance then an appropriate Assurance will be required (See Part III, Section 2.2 Vertebrate Animals for more information).
The five points are as follows:}

\instructions{ \begin{enumerate}
\item Provide a detailed description of the proposed use of the animals in the work outlined in the Research Strategy section. Identify the species, strains, ages, sex, and numbers of animals to be used in the proposed work.
\item Justify the use of animals, the choice of species, and the numbers to be used. If animals are in short supply, costly, or to be used in large numbers, provide an additional rationale for their selection and numbers.
\item Provide information on the veterinary care of the animals involved.
\item Describe the procedures for ensuring that discomfort, distress, pain, and injury will be limited to that which is unavoidable in the conduct of scientifically sound research. Describe the use of analgesic, anesthetic, and tranquilizing drugs and/or comfortable restraining devices, where appropriate, to minimize discomfort, distress, pain, and injury.
\item Describe any method of euthanasia to be used and the reasons for its selection. State whether this method is consistent with the recommendations of the American Veterinary Medical Association (AVMA) Guidelines on Euthanasia. If not, include a scientific justification for not following the recommendations.
\end{enumerate}}

\instructions{ Do not use the vertebrate animal section to circumvent the page limits of the Research Strategy.}
